% Generated by Sphinx.
\documentclass[letterpaper,10pt,english]{howto}
\usepackage[utf8]{inputenc}
\usepackage[T1]{fontenc}
\usepackage{babel}
\usepackage{times}
\usepackage[Sonny]{fncychap}
\usepackage{sphinx}


\title{Module Documentation For pytidylib}
\date{April 07, 2009}
\release{}
\author{Jason Stitt}
\newcommand{\sphinxlogo}{}
\renewcommand{\releasename}{Release}
\makeindex

\newcommand\PYGZat{@}
\newcommand\PYGZlb{[}
\newcommand\PYGZrb{]}
\newcommand\PYGaz[1]{\textcolor[rgb]{0.00,0.63,0.00}{#1}}
\newcommand\PYGax[1]{\textcolor[rgb]{0.84,0.33,0.22}{\textbf{#1}}}
\newcommand\PYGay[1]{\textcolor[rgb]{0.00,0.44,0.13}{\textbf{#1}}}
\newcommand\PYGar[1]{\textcolor[rgb]{0.73,0.38,0.84}{#1}}
\newcommand\PYGas[1]{\textcolor[rgb]{0.25,0.44,0.63}{\textit{#1}}}
\newcommand\PYGap[1]{\textcolor[rgb]{0.00,0.44,0.13}{\textbf{#1}}}
\newcommand\PYGaq[1]{\textcolor[rgb]{0.38,0.68,0.84}{#1}}
\newcommand\PYGav[1]{\textcolor[rgb]{0.00,0.44,0.13}{\textbf{#1}}}
\newcommand\PYGaw[1]{\textcolor[rgb]{0.13,0.50,0.31}{#1}}
\newcommand\PYGat[1]{\textcolor[rgb]{0.73,0.38,0.84}{#1}}
\newcommand\PYGau[1]{\textcolor[rgb]{0.32,0.47,0.09}{#1}}
\newcommand\PYGaj[1]{\textcolor[rgb]{0.00,0.44,0.13}{#1}}
\newcommand\PYGak[1]{\textcolor[rgb]{0.14,0.33,0.53}{#1}}
\newcommand\PYGah[1]{\textcolor[rgb]{0.00,0.13,0.44}{\textbf{#1}}}
\newcommand\PYGai[1]{\textcolor[rgb]{0.73,0.38,0.84}{#1}}
\newcommand\PYGan[1]{\textcolor[rgb]{0.13,0.50,0.31}{#1}}
\newcommand\PYGao[1]{\textcolor[rgb]{0.25,0.44,0.63}{\textbf{#1}}}
\newcommand\PYGal[1]{\textcolor[rgb]{0.00,0.44,0.13}{\textbf{#1}}}
\newcommand\PYGam[1]{\textbf{#1}}
\newcommand\PYGab[1]{\textit{#1}}
\newcommand\PYGac[1]{\textcolor[rgb]{0.25,0.44,0.63}{#1}}
\newcommand\PYGaa[1]{\textcolor[rgb]{0.19,0.19,0.19}{#1}}
\newcommand\PYGaf[1]{\textcolor[rgb]{0.25,0.50,0.56}{\textit{#1}}}
\newcommand\PYGag[1]{\textcolor[rgb]{0.13,0.50,0.31}{#1}}
\newcommand\PYGad[1]{\textcolor[rgb]{0.00,0.25,0.82}{#1}}
\newcommand\PYGae[1]{\textcolor[rgb]{0.13,0.50,0.31}{#1}}
\newcommand\PYGaZ[1]{\textcolor[rgb]{0.25,0.44,0.63}{#1}}
\newcommand\PYGbf[1]{\textcolor[rgb]{0.00,0.44,0.13}{#1}}
\newcommand\PYGaX[1]{\textcolor[rgb]{0.25,0.44,0.63}{#1}}
\newcommand\PYGaY[1]{\textcolor[rgb]{0.00,0.44,0.13}{#1}}
\newcommand\PYGbc[1]{\textcolor[rgb]{0.78,0.36,0.04}{#1}}
\newcommand\PYGbb[1]{\textcolor[rgb]{0.00,0.00,0.50}{\textbf{#1}}}
\newcommand\PYGba[1]{\textcolor[rgb]{0.02,0.16,0.45}{\textbf{#1}}}
\newcommand\PYGaR[1]{\textcolor[rgb]{0.25,0.44,0.63}{#1}}
\newcommand\PYGaS[1]{\textcolor[rgb]{0.13,0.50,0.31}{#1}}
\newcommand\PYGaP[1]{\textcolor[rgb]{0.05,0.52,0.71}{\textbf{#1}}}
\newcommand\PYGaQ[1]{\textcolor[rgb]{0.78,0.36,0.04}{\textbf{#1}}}
\newcommand\PYGaV[1]{\textcolor[rgb]{0.25,0.50,0.56}{\textit{#1}}}
\newcommand\PYGaW[1]{\textcolor[rgb]{0.05,0.52,0.71}{\textbf{#1}}}
\newcommand\PYGaT[1]{\textcolor[rgb]{0.73,0.38,0.84}{#1}}
\newcommand\PYGaU[1]{\textcolor[rgb]{0.13,0.50,0.31}{#1}}
\newcommand\PYGaJ[1]{\textcolor[rgb]{0.56,0.13,0.00}{#1}}
\newcommand\PYGaK[1]{\textcolor[rgb]{0.25,0.44,0.63}{#1}}
\newcommand\PYGaH[1]{\textcolor[rgb]{0.50,0.00,0.50}{\textbf{#1}}}
\newcommand\PYGaI[1]{\fcolorbox[rgb]{1.00,0.00,0.00}{1,1,1}{#1}}
\newcommand\PYGaN[1]{\textcolor[rgb]{0.73,0.73,0.73}{#1}}
\newcommand\PYGaO[1]{\textcolor[rgb]{0.00,0.44,0.13}{#1}}
\newcommand\PYGaL[1]{\textcolor[rgb]{0.02,0.16,0.49}{#1}}
\newcommand\PYGaM[1]{\colorbox[rgb]{1.00,0.94,0.94}{\textcolor[rgb]{0.25,0.50,0.56}{#1}}}
\newcommand\PYGaB[1]{\textcolor[rgb]{0.25,0.44,0.63}{#1}}
\newcommand\PYGaC[1]{\textcolor[rgb]{0.33,0.33,0.33}{\textbf{#1}}}
\newcommand\PYGaA[1]{\textcolor[rgb]{0.00,0.44,0.13}{#1}}
\newcommand\PYGaF[1]{\textcolor[rgb]{0.63,0.00,0.00}{#1}}
\newcommand\PYGaG[1]{\textcolor[rgb]{1.00,0.00,0.00}{#1}}
\newcommand\PYGaD[1]{\textcolor[rgb]{0.00,0.44,0.13}{\textbf{#1}}}
\newcommand\PYGaE[1]{\textcolor[rgb]{0.25,0.50,0.56}{\textit{#1}}}
\newcommand\PYGbg[1]{\textcolor[rgb]{0.44,0.63,0.82}{\textit{#1}}}
\newcommand\PYGbe[1]{\textcolor[rgb]{0.40,0.40,0.40}{#1}}
\newcommand\PYGbd[1]{\textcolor[rgb]{0.25,0.44,0.63}{#1}}
\newcommand\PYGbh[1]{\textcolor[rgb]{0.00,0.44,0.13}{\textbf{#1}}}
\begin{document}

\maketitle
\tableofcontents



The \href{http://countergram.com/software/pytidylib}{pytidylib} module is a Python interface to the \href{http://tidy.sourceforge.net/}{TidyLib} library, which allows you to turn semi-valid HTML or XHTML code into valid code with an HTML 4, XHTML Transitional or XHTML Strict doctype. Some of the library's capabilities include:
\begin{itemize}
\item {} 
Clean up unclosed tags and unescaped characters such as ampersands

\item {} 
Output HTML 4 or XHTML Transitional or Strict

\item {} 
Convert named HTML entities to numeric entities, which are more portable and can be used in XML documents without an HTML doctype

\item {} 
Clean up output from desktop programs such as Word

\item {} 
Indent the output, including proper (i.e. no) indenting for \code{pre} elements

\end{itemize}

The pytidylib package is intended as a replacement for the uTidyLib package. The author found that uTidyLib has not been maintained in a while, and there are several outstanding patches. Compared to uTidyLib, the pytidylib package:
\begin{itemize}
\item {} 
Supports unicode strings

\item {} 
Supports 64-bit systems and OS X

\item {} 
Has improved performance, due to using cStringIO in place of StringIO, having the (optional) ability to re-use document objects, and a few other enhancements

\item {} 
Does not leak memory when used repeatedly, due to proper freeing of document and error-reporting objects

\end{itemize}

This package relies on \code{ctypes}, which was added to Python in version 2.5. For versions 2.3 to 2.4, download \code{ctypes} from the \href{http://python.net/crew/theller/ctypes/}{ctypes home page}.


\section{Installing TidyLib}

You must have \href{http://tidy.sourceforge.net/}{TidyLib} installed to use this Python module. There is no affiliation between the two projects; this is only a quick reference. How best to install \href{http://tidy.sourceforge.net/}{TidyLib} depends on your platform:
\begin{itemize}
\item {} 
Linux/BSD: First, try installing \code{tidylib} (or possibly \code{libtidy}) through your system's package management or ports system.

\item {} 
OS X: You may already have \href{http://tidy.sourceforge.net/}{TidyLib}, especially if you have Apple's Developer Tools installed. In Terminal, run \code{locate libtidy} to find out.

\item {} 
Windows: First, try installing the Windows package from the \href{http://tidy.sourceforge.net/}{TidyLib} homepage. As of this writing, the latest DLL version may not be fully up-to-date.

\item {} 
If none of the above options works, downlaod the source code and build it yourself using the appropriate compiler for your platform. You must download the source using CVS:

\begin{Verbatim}[commandchars=@\[\]]
cvs -z3 -d:pserver:anonymous@PYGZat[]tidy.cvs.sourceforge.net:/cvsroot/tidy co -P tidy
\end{Verbatim}

\end{itemize}


\section{Installing pytidylib}

Use \href{http://peak.telecommunity.com/DevCenter/EasyInstall}{easy\_install}:

\begin{Verbatim}[commandchars=@\[\]]
easy@_install pytidylib
\end{Verbatim}

Or, download the latest pytidylib from SourceForge and install in the usual way:

\begin{Verbatim}[commandchars=@\[\]]
python setup.py install
\end{Verbatim}


\section{Trivial example of use}

The following code cleans up an invalid HTML document and sets an option:

\begin{Verbatim}[commandchars=@\[\]]
@PYGal[from] @PYGaW[tidylib] @PYGal[import] tidy@_document
document, errors @PYGbe[=] tidy@_document(@PYGaB[''']@PYGaB[@textless[]p@textgreater[]f@&otilde;o @textless[]img src=]@PYGaB["]@PYGaB[bar.jpg]@PYGaB["]@PYGaB[@textgreater[]]@PYGaB['''],
    options@PYGbe[=]{@PYGaB[']@PYGaB[numeric-entities]@PYGaB[']:@PYGaw[1]})
@PYGay[print] document
@PYGay[print] errors
\end{Verbatim}


\section{Configuration options}

The Python interface allows you to pass options directly to libtidy. For a complete list of options, see the \href{http://tidy.sourceforge.net/docs/quickref.html}{HTML Tidy Configuration Options Quick Reference} or, from the command line, run \code{tidy -help-config}.

This module sets certain default options, as follows:

\begin{Verbatim}[commandchars=@\[\]]
BASE@_OPTIONS @PYGbe[=] {
    @PYGaB["]@PYGaB[output-xhtml]@PYGaB["]: @PYGaw[1],     @PYGaE[@# XHTML instead of HTML4]
    @PYGaB["]@PYGaB[indent]@PYGaB["]: @PYGaw[1],           @PYGaE[@# Pretty; not too much of a performance hit]
    @PYGaB["]@PYGaB[tidy-mark]@PYGaB["]: @PYGaw[0],        @PYGaE[@# No tidy meta tag in output]
    @PYGaB["]@PYGaB[wrap]@PYGaB["]: @PYGaw[0],             @PYGaE[@# No wrapping]
    @PYGaB["]@PYGaB[alt-text]@PYGaB["]: @PYGaB["]@PYGaB["],        @PYGaE[@# Help ensure validation]
    @PYGaB["]@PYGaB[doctype]@PYGaB["]: @PYGaB[']@PYGaB[strict]@PYGaB['],   @PYGaE[@# Little sense in transitional for tool-generated markup...]
    @PYGaB["]@PYGaB[force-output]@PYGaB["]: @PYGaw[1],     @PYGaE[@# May not get what you expect but you will get something]
    }
\end{Verbatim}

If you do not like these options to be set for you, do the following after importing \code{tidylib}:

\begin{Verbatim}[commandchars=@\[\]]
tidylib@PYGbe[.]BASE@_OPTIONS @PYGbe[=] {}
\end{Verbatim}


\section{Function reference}
\index{tidy\_document() (in module tidylib)}

\hypertarget{tidylib.tidy_document}{}\begin{funcdesc}{tidy\_document}{text, options=None, keep\_doc=False}
Run a string with markup through Tidy and return the entire document.

text (str): The markup, which may be anything from an empty string to a
complete XHTML document. Unicode values are supported; they will be
encoded as utf-8, and tidylib's output will be decoded back to a unicode
object.

options (dict): Options passed directly to tidylib; see the tidylib docs
or run tidy -help-config from the command line.

keep\_doc (boolean): If True, store 1 document object per thread and re-use
it, for a slight performance boost especially when tidying very large numbers
of very short documents.

-\textgreater{} (str, str): The tidied markup {[}0{]} and warning/error messages{[}1{]}.
Warnings and errors are returned just as tidylib returns them.
\end{funcdesc}
\index{tidy\_fragment() (in module tidylib)}

\hypertarget{tidylib.tidy_fragment}{}\begin{funcdesc}{tidy\_fragment}{text, options=None, keep\_doc=False}
Tidy a string with markup and return it without the rest of the document.
Tidy normally returns a full XHTML document; this function returns only
the contents of the \textless{}body\textgreater{} element and is meant to be used for snippets.
Calling tidy\_fragment on elements that don't go in the \textless{}body\textgreater{}, like \textless{}title\textgreater{},
will produce odd behavior.

Arguments and return value as tidy\_document. Note that tidy will always
complain about the lack of a doctype and \textless{}title\textgreater{} element in fragments,
and these errors are not stripped out for you.
\end{funcdesc}
\index{release\_tidy\_doc() (in module tidylib)}

\hypertarget{tidylib.release_tidy_doc}{}\begin{funcdesc}{release\_tidy\_doc}{}
Release the stored document object in the current thread. Only useful
if you have called tidy\_document or tidy\_fragament with keep\_doc=True.
\end{funcdesc}


\renewcommand{\indexname}{Module Index}

\renewcommand{\indexname}{Index}
\printindex
\end{document}
